\documentclass[12pt]{article}
% Encodage et langue
\usepackage[utf8]{inputenc}
\usepackage[T1]{fontenc}
\usepackage[french]{babel}

% Mathématiques
\usepackage{amsmath, amssymb, amsfonts}

% Mise en page
\usepackage[a4paper, margin=2.5cm]{geometry}

% Autres
\usepackage{graphicx}
\usepackage{hyperref}
\hypersetup{colorlinks=true, linkcolor=blue, urlcolor=blue, citecolor=blue}

% Pour éviter les veuves/orphelines
\clubpenalty=10000
\widowpenalty=10000

% En-têtes personnalisés
\usepackage{fancyhdr}
\pagestyle{fancy}
\fancyhead{}
\fancyhead[L]{Année académique 2025--2026}

\begin{document}

\title{Rapport de Simulation}

\author{
    Glineur Pierre \\
    Matricule1 \and
    Tiako Ngouadje Cardin Patson \\
    Matricule2
    }
\date{\today}
\maketitle

\tableofcontents
\newpage

\section{Exécution}
Cette section détaille la structure du projet et la marche à suivre pour reproduire nos résultats.

\subsection{Structure du projet}
Le projet est organisé autour d'un moteur de simulation discret (Simulator.py) et d'entités réseau (Host.py, Router.py, NIC.py, Link.py).

\subsection{Instructions de reproduction}
Pour reproduire les scénarios décrits dans ce rapport, exécutez les scripts suivants à la racine du projet :
\begin{itemize}
    \item Mode 1 : python mode1NO\_RELIABILITY.py
    \item Mode 2 : python mode2acquittements.py
    \item Mode 3 : python mode3avecretransmission.py
    \item Mode 4 : python mode4pipelining.py
    \item Mode 5 : python mode5pipeliningavecfenetredynamique.py
    \item Analyse Globale : python run\_all\_scenarios.py (Script maître avec paramètres de perte éditables).
\end{itemize}

\section{Implémentation}
\subsection{Approche choisie}
Notre implémentation repose sur l'extension de la classe Host pour gérer les mécanismes de fiabilité décrits dans le chapitre 3 du cours (Transport Layer). Nous avons mis en place :
\begin{itemize}
    \item Le Frigo (\_unacked\_packets) : Gestion de la fenêtre glissante.
    \item Le Buffer (\_buffer) : File d'attente logicielle pour le stockage des paquets en attente de fenêtre libre.
    \item Le Timer unique : Implémentation du protocole rdt 3.0 avec un décompte pour le paquet le plus ancien.
\end{itemize}

\subsection{Difficultés rencontrées}
\begin{itemize}
    \item Gestion des ACKs cumulatifs : S'assurer que la réception d'un ACK libère correctement tous les paquets précédents du "frigo".
    \item Synchronisation du Timer : Éviter les redondances de timers lors de l'envoi rapide de plusieurs paquets en mode Pipelining.
\end{itemize}

\subsection{État final}
L'implémentation supporte les 5 modes demandés. Les modes 2 et 3 fonctionnent en Stop-and-Wait, tandis que les modes 4 et 5 exploitent le Pipelining pour optimiser l'utilisation de la bande passante.

\section{Simulations}
\subsection{Analyse des résultats et réponses aux questions de l'énoncé}
\subsubsection{Stop and Wait (Modes 2 \& 3)}
\begin{itemize}
    \item Mode 2 (ACK) : Sans timer, nous observons que la simulation se bloque indéfiniment en cas de perte (Lien L1 avec lost\_prob \> 0). Cela illustre les limites du protocole rdt 2.2 décrit aux pages 34-40 du cours.
    \item Mode 3 (Retransmission) : Le log montre @0.500000 Timer expired! Retransmitting. L'émetteur détecte la perte et survit à l'incident.
    \item Temps moyen de transmission : [Insérez votre calcul ici : Moyenne des temps entre sends et received ACK].
\end{itemize}
\subsubsection{Pipelining avec fenêtre fixe (Mode 4)}
\begin{itemize}
    \item Observations : Contrairement au mode 3, l'émetteur envoie plusieurs paquets (fenêtre de 4) dès t=0.000000 sans attendre les premiers ACKs.
    \item Comparaison : Le temps moyen de transmission diminue drastiquement car le lien ne reste pas inactif durant le temps de propagation (RTT).
\end{itemize}
\subsubsection{Pipelining avec fenêtre dynamique (Mode 5)}
\begin{itemize}
    \item Comportement attendu : La fenêtre commence à 1, augmente à chaque ACK, et chute à 1 en cas de timeout.
    \item Analyse du Goulot : Avec un lien L2 à 500 kbps et une file au routeur de 10, nous observons une congestion.
    \item Débit moyen effectif : il faudra Insérer le calcul : (Paquets * Taille * 8) / Temps total.
    \item Justification : L'ajustement dynamique permet de trouver le débit maximal du lien critique (500 kbps) sans saturer indéfiniment le routeur.
\end{itemize}
\section{Conclusion}
La comparaison entre les modes fixe et dynamique montre que le mode dynamique (inspiré de TCP) est plus résilient et efficace pour s'adapter aux capacités variables des nœuds intermédiaires (routeurs).
\end{document}